\documentclass[9.5pt]{beamer}

\mode<presentation>
{
  \usetheme{Warsaw}       % or try default, Darmstadt, Warsaw, ...
  \usecolortheme{default} % or try albatross, beaver, crane, ...
  \usefonttheme{serif}    % or try default, structurebold, ...
  \setbeamertemplate{navigation symbols}{}
  \setbeamertemplate{caption}[numbered]
} 

\usepackage[utf8]{inputenc} % accents 8 bits dans le fichier
\usepackage[T1]{fontenc}      % accents codés dans la fonte
\usepackage[french]{babel}
\usepackage{amsmath,amssymb}
\usepackage{graphicx}
\usepackage{fancyhdr}
\usepackage{siunitx}
\usepackage[mode=buildnew]{standalone}
\usepackage{algorithmicx}
\usepackage{algorithm}
\usepackage{algpseudocode}
\usepackage{pgfplots}

\pgfplotsset{compat=1.3}

\pgfplotsset{
    discard if not/.style 2 args={
        x filter/.code={
            \edef\tempa{\thisrow{#1}}
            \edef\tempb{#2}
            \ifx\tempa\tempb
            \else
                \def\pgfmathresult{inf}
            \fi
        }
    }
}

% Here's where the presentation starts, with the info for the title slide
\title[MPNA : MIS]{Présentation AOC \\Application MiniMD}
\author[\bsc{Beaupère} \& \bsc{Granger}]{Matthias \bsc{Beaupère} \& Pierre \bsc{Granger}}
\institute{M2 CHPS}
\date{\today}

\begin{document}
\setbeamercolor{captioncolor}{fg=white,bg=red!80!white}
\setbeamertemplate{caption}{%
\begin{beamercolorbox}[wd=0.8\linewidth, sep=.2ex]{captioncolor}\tiny\centering\insertcaption%
\end{beamercolorbox}%
}

\begin{frame}
  \titlepage
\end{frame}

\begin{frame}{Plan}
	\tableofcontents[hideallsubsections]
\end{frame}

\section{Présentation de la dynamique moléculaire}
	\begin{frame}{Présentation de la dynamique moléculaire}
		\begin{block}{Dynamique moléculaire}
			Simulation numérique de l'intéraction des atomes entre eux.
			\begin{itemize}
				\item Processus iteratif
				\item Intégration des équations du mouvement
				\item Potentiels d'interaction entre les atomes
				\item Exemple d'utilisation : repliement des protéines
			\end{itemize}
		\end{block}

		\begin{figure}
			\includegraphics[width=0.25\linewidth]{images/bs_md.png}
		\end{figure}
	\end{frame}

\section{Présentation de l'application}
	\begin{frame}{Présentation de l'application}
		\begin{block}{Caractéristiques techniques}
			\begin{itemize}
				\item 5000 ligne de code C++ parallèle
				\item Utilise MPI et OpenMP
				\item Seulement 2 potentiels implémentés
				\item Nombreux paramètres simulables
			\end{itemize}
		\end{block}

		\begin{block}{Objectif}
			Programme de test des architectures parallèle en modélisant simplement le comportant d'un code de simulation de dynamique moléculaire.
		\end{block}
	\end{frame}

\section{Etude de performances}
	\subsection{Cadre d'expérimentation}
		\begin{frame}{Matériel}
			\begin{block}{Processeur et mémoire vive}
				\begin{itemize}
					\item Intel Core i5-760 2.80GHz (2011)
					\item 2 c\oe{}urs physiques (4 logiques)
					\item 8Go DDR3
				\end{itemize}
			\end{block}

			\begin{block}{Micro-architecture}
				\begin{itemize}
					\item Nehalem
					\item SSE 4.2 $\rightarrow$ vecto sur 128 bits
				\end{itemize}
			\end{block}
		\end{frame}

		\begin{frame}{Compilation}
			\begin{block}{Compilateur et options}
				\begin{itemize}
					\item GCC 5.4
					\item \verb!-Ofast!
					\item \verb!--funroll-loops!
					\item \verb!-g!
				\end{itemize}
			\end{block}
		\end{frame}

		\begin{frame}{Exécution et profiling}
			\begin{block}{Paramétrage de miniMD}
				\begin{itemize}
					\item Taille du problème 64 x 64 x 64
					\item Densité 0.8 $\rightarrow 10^6$ atomes
					\item 4 threads OpenMP
					\item 1 processus MPI
					\item timestep 5 ms
					\item durée de simulation 0.5s
				\end{itemize}
			\end{block}
		\end{frame}

	\subsection{Résultats}
		\begin{frame}{Etude générale}
			
			\begin{figure}[h!]
				\centering
				\begin{center}
					\includegraphics[width=200px]{images/maqao_global_metrics.png}
					\caption{Métriques global obtenu avec Maqao pour 4 threads}
					\label{global_maqao}
				\end{center}
			\end{figure}
		\end{frame}

		\begin{frame}{Etude générale}
			
			\begin{figure}[h!]
				\centering
				\begin{center}
					\includegraphics[width=300px]{images/maqao_loops.png}
					\caption{Statistiques des boucles obtenu avec Maqao pour 4 threads}
					\label{loop_maqao}
				\end{center}
			\end{figure}
		\end{frame}

		\begin{frame}[fragile]{Analyse des points chauds}
			\begin{block}{3 instructions - 45\% du temps d'exécution}
				\begin{figure}[h!]
					\begin{verbatim}
						#pragma omp atomic
						f[j * PAD + 0] -= delx * force;
						#pragma omp atomic
						f[j * PAD + 1] -= dely * force;
						#pragma omp atomic
						f[j * PAD + 2] -= delz * force;
					\end{verbatim}
					\centering
					\caption{Extrait de code correspondant au point chaud}
					\label{code:hotspot}
				\end{figure}
			\end{block}
\end{frame}

		\begin{frame}[fragile]{Analyse des points chauds}
			\begin{block}{Code assembleur d'une instruction}
				\begin{figure}[h!]
					\begin{verbatim}
						0x416590 MOV %RAX,%RDX
						0x416593 MOV %RDX,0x8(%RSP)
						0x416598 MOV %RDX,%RAX
						0x41659b MOVSD 0x8(%RSP),%XMM2
						0x4165a1 SUBSD %XMM4,%XMM2
						0x4165a5 MOVQ %XMM2,%R8
						0x4165aa LOCK CMPXCHG %R8,(%RCX)
						0x4165af CMP %RAX,%RDX
						0x4165b2 JNE 416590
					\end{verbatim}
					\centering
					\caption{Compilation d'un statement du point chaud}
					\label{code:hotspot}
				\end{figure}
			\end{block}
\end{frame}

\section{Optimisations}
	\subsection{Tentatives d'optimisation}
		\begin{frame}[fragile]{Tentatives d'optimisation}
			\begin{block}{Section critique}
				\scriptsize
				\begin{verbatim}
					#pragma omp critical
					{
						f[j * PAD + 0] -= delx * force;
						f[j * PAD + 1] -= dely * force;
						f[j * PAD + 2] -= delz * force;
					}
				\end{verbatim}
			\end{block}

			\begin{block}{Mémoire transactionnelle}
				\scriptsize
				\begin{verbatim}
					__transaction_atomic
					{
						f[j * PAD + 0] -= delx * force;
						f[j * PAD + 1] -= dely * force;
						f[j * PAD + 2] -= delz * force;
					}
				\end{verbatim}
			\end{block}
\end{frame}

	\subsection{Optimisations envisageables}
		\begin{frame}{Vectorisation explicite}
			\begin{columns}
				\column{0.7\linewidth}
					\begin{figure}
						\includegraphics[width=\linewidth]{images/vecto.png}
					\end{figure}
				\column{0.3\linewidth}
					\begin{block}{Algorithme}
						\begin{itemize}
							\item Gather AoS
							\item AoS $\longrightarrow$ SoA
							\item Calculs
							\item SoA $\longrightarrow$ AoS
							\item Scatter AoS
						\end{itemize}
					\end{block}
			\end{columns}
		\end{frame}

\section{Analyse du parallélisme}
	\begin{frame}{Analyse du parallélisme}
		\begin{figure}
			\centering
			\begin{tikzpicture}[scale=0.8]
				\begin{axis}[minor tick num=1,
					xlabel=Nombre total de threads,
					ylabel=Temps d'exécution (\si{\second}),
					ymin=0,
					ymax=66]
				\addplot [green, mark=*, discard if not={MPI}{1}] table [x={OMP}, y={t}] {../rapport/data.res};
				\addplot [blue, mark=*, discard if not={MPI}{2}] table [x expr=\thisrowno{1}*2, y={t}] {../rapport/data.res};
				\addplot [red, mark=*, discard if not={MPI}{4}] table [x expr=\thisrowno{1}*4, y={t}] {../rapport/data.res};
				\draw [dashed] (axis cs:9, \pgfkeysvalueof{/pgfplots/ymin}) -- (axis cs:9, \pgfkeysvalueof{/pgfplots/ymax});
				\legend{MPI = 1, MPI = 2, MPI = 4}
				\addlegendimage{dashed}
				\addlegendentry{Hyperthreading}
				\end{axis}
			\end{tikzpicture}
			\caption{Evolution du temps d'exécution en fonction du nombre de processus MPI et du nombre total de threads.\label{fig:tvsparallel}}
		\end{figure}
	\end{frame}

\section{Conclusion}
	\begin{frame}{Conclusion}
		\begin{block}{Perfomances}
			\begin{itemize}
				\item L'application se parallélise bien, peu de communications
				\item Bons flot de contrôle et accès à la mémoire
				\item Un point chaud majeur au niveau du noyau de calcul
			\end{itemize}
		\end{block}

		\begin{alertblock}{Optimisations du point chaud}
			\begin{itemize}
				\item Optimisations limitées à cause des opérations atomiques
				\item Possibilité d'optimiser les accès avec des conversions AoS/SoA.
				\item Changer l'algorithme afin d'enlever les instructions atomiques : 2 threads ne doivent pas modifier la même donnée.

			\end{itemize} 
		\end{alertblock}
	\end{frame}

\end{document}
