\documentclass[11pt,a4paper]{article}


\setlength{\topmargin}{-55pt}%
\setlength{\oddsidemargin}{-20pt}%
\setlength{\textwidth}{490pt}%
\setlength{\textheight}{700pt}%
\setlength{\headsep}{20pt}%
\setlength{\headheight}{14pt}

\usepackage[utf8]{inputenc} % accents 8 bits dans le fichier
\usepackage[T1]{fontenc}      % accents codés dans la fonte
\usepackage[french]{babel}
\usepackage{amsmath,amssymb}
\usepackage{graphicx}
\usepackage{fancyhdr}
\usepackage{booktabs}
\usepackage{color, colortbl}
\usepackage{appendix}
\usepackage{pgfplots}
\usepackage[hidelinks]{hyperref}
\usepackage{siunitx}
\usepackage{subcaption}

\pgfplotsset{compat=1.3}

\addto\captionsfrench{% Replace "english" with the language you use
  \renewcommand{\contentsname}%
    {Table des matières}
}

\DecimalMathComma

\lhead{}      %en-tête
\chead{AOC : Application MiniMD}%
\rhead{}%
\lfoot{\tiny{Pierre GRANGER \& Matthias BEAUPERE}}
\cfoot{}%
\rfoot{\thepage}%
\renewcommand{\headrulewidth}{0.5pt}
\renewcommand{\footrulewidth}{0.5pt}
\pagestyle{fancy}

\newcommand{\HRule}{\rule{\linewidth}{0.5mm}}
\newcommand{\norm}[1]{\left\lVert#1\right\rVert}

\definecolor{green}{rgb}{0.2,0.8,0.2}

\begin{document}
\begin{center}

	{\LARGE\centering Projet d'AOC :\\ Application de test MiniMD}\\[1cm]

	{ Matthias \bsc{Beaupère}, Pierre \bsc{Granger}}\\[0.5cm]
	{Rapport AOC - CHPS - \today}\\[2cm]
\end{center}

\tableofcontents
\newpage

\section{Présentation de l'application}
	L'application sur laquelle nous avons travaillé est MiniMD. Cette application de simulation en dynamique moléculaire est écrite en C++ et est parallèle. Son objectif est de pouvoir tester l'efficacité des nouvelles architectures parallèles dans l'exécution de codes de simulation de dynamique moléculaire. Dans cet objectif, cette mini-application a été conçue afin d'être simple et légère ainsi que facilement adaptable sur de nouvelles architectures matérielles. Les algorithmes de base utilisés dans ce code suivent les algorithmes d'un code de dynamique moléculaire de production bien plus important, LAMMPS.

	L'application est écrite en C++ et comporte moins de 5000 lignes de code. Elle permet d'effectuer des simulations de dynamique moléculaire avec des potentiels de Lennard-Jones ou des modèles EAM (Emnbedded Atom Model) uniquement. La parallélisation des calculs est effectuée via une décomposition spatiale du domaine d'étude : chaque processeur du cluster effectue ses calculs sur une partie du domaine de simulation. Le code est sensé passer correctement à l'échelle  sur une architecture parallèle. L'application peut fonctionner avec plusieurs librairies de parallélisation proposant des types de paralllisme différents : MPI, OpenMP, OpenCL, Kokkos, OpenACC, ... Dans ce rapport, seule l'implémentation utilisant MPI pour le prallélisme en mémoire distribuée et OpenMP pour le parallélisme en mémoire partagée sera étudiée.

	Comme dans des codes de simulations plus complexes, miniMD permet à l'utilisateur de spécifier de nombreux paramètres du problèmes à étudier : la taille du problème, la température, la densité atomique, la durée des pas de temps, le nombre de pas de temps à simuler et la distance de coupure à utiliser pour l'intéraction considérée. Néanmoins, seuls deux types d'interactions sont disponibles (Lennard-Jones et EAM). En effet, les intéractions électrostatiques à longue distance ainsi que les champs de force moléculaires ne peuvent pas être simulés avec cette application. L'ajout de telles fonctionnalités n'aurait fait que compliquer le code alors qu'elles ne sont pas nécessaires afin d'effectuer des tests avec de la dynamique moléculaire basique.
	
\section{Bilan de performances}

\section{Optimisations}

\end{document}
